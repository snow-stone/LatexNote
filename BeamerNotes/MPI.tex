%%%%%%%%%%%%%%%%%%%%%%%%%%%%%%%%%%%%%%%%%
% Beamer Presentation
% LaTeX Template
% Version 1.0 (10/11/12)
%
% This template has been downloaded from:
% http://www.LaTeXTemplates.com
%
% License:
% CC BY-NC-SA 3.0 (http://creativecommons.org/licenses/by-nc-sa/3.0/)
%
%%%%%%%%%%%%%%%%%%%%%%%%%%%%%%%%%%%%%%%%%

%----------------------------------------------------------------------------------------
%	PACKAGES AND THEMES
%----------------------------------------------------------------------------------------

\documentclass{beamer}

\mode<presentation> {

% The Beamer class comes with a number of default slide themes
% which change the colors and layouts of slides. Below this is a list
% of all the themes, uncomment each in turn to see what they look like.

%\usetheme{default}
%\usetheme{AnnArbor}
%\usetheme{Antibes}
%\usetheme{Bergen}
\usetheme{Berkeley}
%\usetheme{Berlin}
%\usetheme{Boadilla}
%\usetheme{CambridgeUS}
%\usetheme{Copenhagen}
%\usetheme{Darmstadt}
%\usetheme{Dresden}
%\usetheme{Frankfurt}
%\usetheme{Goettingen}
%\usetheme{Hannover}
%\usetheme{Ilmenau}
%\usetheme{JuanLesPins}
%\usetheme{Luebeck}
%\usetheme{Madrid}
%\usetheme{Malmoe}
%\usetheme{Marburg}
%\usetheme{Montpellier}
%\usetheme{PaloAlto}
%\usetheme{Pittsburgh}
%\usetheme{Rochester}
%\usetheme{Singapore}
%\usetheme{Szeged}
%\usetheme{Warsaw}

% As well as themes, the Beamer class has a number of color themes
% for any slide theme. Uncomment each of these in turn to see how it
% changes the colors of your current slide theme.

%\usecolortheme{albatross}
%\usecolortheme{beaver}
%\usecolortheme{beetle}
%\usecolortheme{crane}
%\usecolortheme{dolphin}
%\usecolortheme{dove}
%\usecolortheme{fly}
%\usecolortheme{lily}
%\usecolortheme{orchid}
%\usecolortheme{rose}
%\usecolortheme{seagull}
%\usecolortheme{seahorse}
%\usecolortheme{whale}
%\usecolortheme{wolverine}

%\setbeamertemplate{footline} % To remove the footer line in all slides uncomment this line
%\setbeamertemplate{footline}[page number] % To replace the footer line in all slides with a simple slide count uncomment this line

%\setbeamertemplate{navigation symbols}{} % To remove the navigation symbols from the bottom of all slides uncomment this line
}

\usepackage{graphicx} % Allows including images
\usepackage{booktabs} % Allows the use of \toprule, \midrule and \bottomrule in tables
\usepackage{lmodern}

%----------------------------------------------------------------------------------------
%	TITLE PAGE
%----------------------------------------------------------------------------------------

\title[Short title]{Short note on MPI} % The short title appears at the bottom of every slide, the full title is only on the title page

\author{Haining LUO} % Your name
\institute[] % Your institution as it will appear on the bottom of every slide, may be shorthand to save space
{
ECL \\ % Your institution for the title page
\medskip
\textit{haining.luo@doctorant.ec-lyon.fr} % Your email address
}
\date{\today} % Date, can be changed to a custom date

\begin{document}

\begin{frame}
\titlepage % Print the title page as the first slide
\end{frame}

\begin{frame}
\frametitle{Overview} % Table of contents slide, comment this block out to remove it
\tableofcontents % Throughout your presentation, if you choose to use \section{} and \subsection{} commands, these will automatically be printed on this slide as an overview of your presentation
\end{frame}

%----------------------------------------------------------------------------------------
%	PRESENTATION SLIDES
%----------------------------------------------------------------------------------------
\section{MPI} % Sections can be created in order to organize your presentation into discrete blocks, all sections and subsections are automatically printed in the table of contents as an overview of the talk
%------------------------------------------------

\subsection{MPI in general} % A subsection can be created just before a set of slides with a common theme to further break down your presentation into chunks

\begin{frame}
\frametitle{MPI setup}

set env variable for intel parallel studio : 
source /opt/intel/compilers\_and\_libraries\_2016.3.210/linux/bin/compilervars.sh intel64 \\

hluo@mint ~\$ which ifort \\
/opt/intel/compilers\_and\_libraries\_2016.3.210/linux/bin/intel64/ifort \\

mpif90 is a wapper :  mpif90 will call gfortran to compile fortran source with MPI lib \\

whereas ifort is just the compiler, need explicit option -L -l to load mpi lib. \\

mpiifort is the wapper for intel fortran compiler.

\end{frame}

%------------------------------------------------

\begin{frame}
\frametitle{MPI wrappers}
\begin{itemize}
	\item mpif90 mpif77 for gfortran

	\item mpicc mpicxx for gcc g++

	\item mpiifort mpiicc mpiicpc for ifort icc icpc

	\item mpirun invokes (or equals) \\
mpdboot mpiexec and mpdallexit \\
Use "mpirun" when you don not need MPD daemons after the program run. \\
Using mpirun is the recommended practice when using a resource manager, such as PBS Pro* or LSF*
\end{itemize}
\end{frame}
%------------------------------------------------

\begin{frame}
\frametitle{MPI implementation and debug tip}
\begin{itemize}
	\item MPICH2 respect absolutely the consigne.
	\item Open MPI just has some different ideas, not following exactly the protocal to every line, those (not necessarily encounter often) can be tricky.
	\item When doing MPI-OpenMP hybrid, often (when you are experienced with MPI) it's OpenMP that works not. Try running the code on different machines and with different versions of MPI-OpenMP. Ca peut aider a trouver des problem qui bloque beaucoup !
\end{itemize}
\end{frame}

%------------------------------------------------

\begin{frame}
	\frametitle{MPI headfiles \& difference between C and Fortran}

	Every program who make call sub-program MPI must include a head file.
\begin{block}{Fortran}
	MPI-2 : use mpi	\\
	MPI-1 : include the file mpif.h
\end{block}
\begin{block}{C/C++}
	Must include the file mpi.h
	except for MPI\_INIT, arguments of the MPI sub-program are the same as Fortran.
\end{block}

\begin{block}{style of C and Fortran}
	Fortran :
	integer, intent(out) :: code \\
	call MPI\_INIT(code) 

	integer, intent(out) :: code \\
	call MPI\_FINALIZE(code) 
	C : 
	int MPI\_Init(int *argc, char ***argv); \\
	int MPI\_Finalize(void);
\end{block}
	
\end{frame}
%------------------------------------------------

\begin{frame}
\frametitle{Tools with parallel programming}
\begin{block}{debug}
Totalview and DDT
\end{block}

\begin{block}{visualize parallel computing}
MPE (MPI Parallel Environment), FPMPI and Scalasca (with graphic interface)
\end{block}

\begin{block}{jumpshot and traceAnalyzer}
		jumpshot comes with mpich
		traceAnalyzer comes with intel compiler
\end{block}

\end{frame}

%------------------------------------------------
\begin{frame}
\frametitle{parallelized library}
\begin{block}{lib}
ScaLAPACK : direct resolution of linear algebra \\
PETSc     : iterative method of linear and no-linear system \\
PaStiX    : resolution of parse matrix \\
FFTW      : fast fourier transform 
\end{block}

\end{frame}

%------------------------------------------------
\begin{frame}
\frametitle{A word on openmpi}

After openmpi's installation, in the binaries beside mpirun, mpif90 there is also ompi\_info

ompi\_info | grep -i slurm \\
MCA ras: slurm (MCA v2.0, API v2.0, Component v1.6.5) \\
MCA plm: slurm (MCA v2.0, API v2.0, Component v1.6.5) \\ 
MCA ess: slurm (MCA v2.0, API v2.0, Component v1.6.5) \\
MCA ess: slurmd (MCA v2.0, API v2.0, Component v1.6.5) 

On Newton, submission of the job is done by slurm. Openmpi seems then well configured.

Installation of foam-extend-4.0 on newton, requires an extra line in /etc/presh

\end{frame}




%------------------------------------------------
\begin{frame}
\frametitle{Multiple Columns}
\begin{columns}[c] % The "c" option specifies centered vertical alignment while the "t" option is used for top vertical alignment

\column{.45\textwidth} % Left column and width
\textbf{Heading}
\begin{enumerate}
\item Statement
\item Explanation
\item Example
\end{enumerate}

\column{.5\textwidth} % Right column and width
Lorem ipsum dolor sit amet, consectetur adipiscing elit. Integer lectus nisl, ultricies in feugiat rutrum, porttitor sit amet augue. Aliquam ut tortor mauris. Sed volutpat ante purus, quis accumsan dolor.

\end{columns}
\end{frame}


\begin{frame}
\frametitle{Theorem}
\begin{theorem}[Mass--energy equivalence]
$E = mc^2$
\end{theorem}
\end{frame}

%------------------------------------------------

\begin{frame}
\frametitle{Theorem}
\begin{theorem}[Mass--energy equivalence]
$E = mc^2$
\end{theorem}
\end{frame}

%------------------------------------------------

\begin{frame}
\frametitle{Table}
\begin{table}
\begin{tabular}{l l l}
\toprule
\textbf{Treatments} & \textbf{Response 1} & \textbf{Response 2}\\
\midrule
Treatment 1 & 0.0003262 & 0.562 \\
Treatment 2 & 0.0015681 & 0.910 \\
Treatment 3 & 0.0009271 & 0.296 \\
\bottomrule
\end{tabular}
\caption{Table caption}
\end{table}
\end{frame}

%------------------------------------------------

\begin{frame}[fragile] % Need to use the fragile option when verbatim is used in the slide
\frametitle{Verbatim}
\begin{example}[Theorem Slide Code]
\begin{verbatim}
\begin{frame}
\frametitle{Theorem}
\begin{theorem}[Mass--energy equivalence]
$E = mc^2$
\end{theorem}
\end{frame}\end{verbatim}
\end{example}
\end{frame}

%------------------------------------------------

\begin{frame}
\frametitle{Figure}
Uncomment the code on this slide to include your own image from the same directory as the template .TeX file.
%\begin{figure}
%\includegraphics[width=0.8\linewidth]{test}
%\end{figure}
\end{frame}

%------------------------------------------------

\begin{frame}[fragile] % Need to use the fragile option when verbatim is used in the slide
\frametitle{Citation}
An example of the \verb|\cite| command to cite within the presentation:\\~

This statement requires citation \cite{p1}.
\end{frame}

%------------------------------------------------

\begin{frame}
\frametitle{References}
\footnotesize{
\begin{thebibliography}{99} % Beamer does not support BibTeX so references must be inserted manually as below
\bibitem[Smith, 2012]{p1} John Smith (2012)
\newblock Title of the publication
\newblock \emph{Journal Name} 12(3), 45 -- 678.
\end{thebibliography}
}
\end{frame}

%------------------------------------------------

\begin{frame}
\Huge{\centerline{The End}}
\end{frame}

%----------------------------------------------------------------------------------------

\end{document} 
