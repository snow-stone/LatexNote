%%%%%%%%%%%%%%%%%%%%%%%%%%%%%%%%%%%%%%%%%
% Beamer Presentation
% LaTeX Template
% Version 1.0 (10/11/12)
%
% This template has been downloaded from:
% http://www.LaTeXTemplates.com
%
% License:
% CC BY-NC-SA 3.0 (http://creativecommons.org/licenses/by-nc-sa/3.0/)
%
%%%%%%%%%%%%%%%%%%%%%%%%%%%%%%%%%%%%%%%%%

%----------------------------------------------------------------------------------------
%	PACKAGES AND THEMES
%----------------------------------------------------------------------------------------

\documentclass{beamer}

\mode<presentation> {

% The Beamer class comes with a number of default slide themes
% which change the colors and layouts of slides. Below this is a list
% of all the themes, uncomment each in turn to see what they look like.

%\usetheme{default}
%\usetheme{AnnArbor}
%\usetheme{Antibes}
%\usetheme{Bergen}
\usetheme{Berkeley}
%\usetheme{Berlin}
%\usetheme{Boadilla}
%\usetheme{CambridgeUS}
%\usetheme{Copenhagen}
%\usetheme{Darmstadt}
%\usetheme{Dresden}
%\usetheme{Frankfurt}
%\usetheme{Goettingen}
%\usetheme{Hannover}
%\usetheme{Ilmenau}
%\usetheme{JuanLesPins}
%\usetheme{Luebeck}
%\usetheme{Madrid}
%\usetheme{Malmoe}
%\usetheme{Marburg}
%\usetheme{Montpellier}
%\usetheme{PaloAlto}
%\usetheme{Pittsburgh}
%\usetheme{Rochester}
%\usetheme{Singapore}
%\usetheme{Szeged}
%\usetheme{Warsaw}

% As well as themes, the Beamer class has a number of color themes
% for any slide theme. Uncomment each of these in turn to see how it
% changes the colors of your current slide theme.

%\usecolortheme{albatross}
%\usecolortheme{beaver}
%\usecolortheme{beetle}
%\usecolortheme{crane}
%\usecolortheme{dolphin}
%\usecolortheme{dove}
%\usecolortheme{fly}
%\usecolortheme{lily}
%\usecolortheme{orchid}
%\usecolortheme{rose}
%\usecolortheme{seagull}
%\usecolortheme{seahorse}
%\usecolortheme{whale}
%\usecolortheme{wolverine}

%\setbeamertemplate{footline} % To remove the footer line in all slides uncomment this line
%\setbeamertemplate{footline}[page number] % To replace the footer line in all slides with a simple slide count uncomment this line

%\setbeamertemplate{navigation symbols}{} % To remove the navigation symbols from the bottom of all slides uncomment this line
}

\usepackage{graphicx} % Allows including images
\usepackage{booktabs} % Allows the use of \toprule, \midrule and \bottomrule in tables
\usepackage{lmodern}

%----------------------------------------------------------------------------------------
%	TITLE PAGE
%----------------------------------------------------------------------------------------

\title[Notes]{Efficiency} % The short title appears at the bottom of every slide, the full title is only on the title page

\author{Haining LUO} % Your name
\institute[] % Your institution as it will appear on the bottom of every slide, may be shorthand to save space
{
ECL \\ % Your institution for the title page
\medskip
\textit{haining.luo@doctorant.ec-lyon.fr} % Your email address
}
\date{\today} % Date, can be changed to a custom date

\begin{document}

\begin{frame}
\titlepage % Print the title page as the first slide
\end{frame}

\begin{frame}
\frametitle{Overview} % Table of contents slide, comment this block out to remove it
\tableofcontents % Throughout your presentation, if you choose to use \section{} and \subsection{} commands, these will automatically be printed on this slide as an overview of your presentation
\end{frame}

%----------------------------------------------------------------------------------------
%	PRESENTATION SLIDES
%----------------------------------------------------------------------------------------
\section{work efficiency}

%----------------------------------------------------------------------------------------

\begin{frame}
\frametitle{Bash win10}
\begin{itemize}

	\item for graphic interface, install Ximg on windows, run it. And lance in the bash linux sub-os terminal with command : \$DISPLAY=:0 texmaker
	
	\item failed to get vpn work on linux sub-os. Remedy : get vpn to work on windows.

\end{itemize}
\end{frame}

%------------------------------------------------

\begin{frame}
\frametitle{Latex linux}
\begin{itemize}

	\item \$ sudo apt-get install texmaker
	
	\item \$ sudo apt-get install latex-beamer
	
	\item for the language package used by Pauly : \$ sudo apt-get install tex-live-lang-french  
	
	\item for other fonts when using templates of journals: \$ sudo apt-get install texlive-fonts-recommended

	\item generate pdf from *.tex : \$ pdflatex *.tex

\end{itemize}
\end{frame}

%------------------------------------------------

\begin{frame}
\frametitle{EasyBuild}
\begin{itemize}

	\item   plateform and env

			Ubuntu 14.04.5 LTS trusty (win10 sub-system)\\
			gcc (system)\\
			g++ (system)\\
			anaconda-1.4.0 (for python)\\
			lua-5.1.4.8\\
			Lmod-6.1
			
	\item	steps \\
			install anaconda \\
			install lua ;
			export PATH to $.../lua/bin$ \\
			install Lmod ;
			export PATH to $.../lmod/6.1/libexec$ ;
			source $.../lmod/6.1/init/bash$ ;
			$export \ LMOD\_CMD=.../lmod/6.1/libexec/lmod$\\
			python $bootstrap\_eb.py$ PrefixofEasyBuild

\end{itemize}
\end{frame}

%------------------------------------------------

\begin{frame}
\frametitle{Easybuild config}
\begin{itemize}

	\item In config file $/home/hluo/.config/easybuild/config.cfg$
	you can then specify : \\
	buildpath = $/tmp/dir$ \\
	installpath=$/home/hluo/software/dir$\\
	etc...\\
	and check by \$ eb --show-config (after \$ module load EasyBuild)
	
	\item for \$ module load EasyBuild to work
	
	need to make sure that "module knows where easybuild is" by using "\$ module use $/home/hluo/software/easybuild/modules/all$" or by env variable \$MODULEPATH \\
	
	when seperating installpath with the installpath of easybuild, after installation by eb you need to have the installpath known by modules.

\end{itemize}
\end{frame}

%------------------------------------------------

\begin{frame}
\frametitle{Newton}
\begin{itemize}

	\item sinfo : info des neouds

	\item	sinfo | grep idle : available neouds.

	\item	scontrol show reservation 

	\item	sprio : show priority of all the job in queue.

	\item	squeue :

	\item	module show icc/2015.1.133-GCC-4.9.2

	\item sudo update-alternatives --config gcc

\end{itemize}
\end{frame}

%------------------------------------------------

\begin{frame}
\frametitle{visu}
	configure the x2goclient \\
	connect visu \\
	use another command to call paraview : vglrun paraview \\
	
	and it works well even in my room!
\end{frame}

%------------------------------------------------

\begin{frame}
\frametitle{Pointwise}
	bash pw-V17.3R5-linux\_x86\_64-jre.sh -c \\
	-c will do the installation with consel/terminal instead of a GUI written by Java, has encountered problem with compatibility of jre
\end{frame}

%------------------------------------------------

\begin{frame}
\frametitle{Inkscape}
	add extension for latex formular input:\\
	\$ sudo apt-get install texlive pstoedit \\
	it will then be : Extensions$->$Render$->$LaTeX Formula
\end{frame}

%------------------------------------------------


\begin{frame}
\frametitle{ssh alias}
\begin{itemize}
	\item Creat alias for remote machine \\
	      Edit \~\thinspace .ssh/config (if there is none, creat one) like this :\\
	      Host zaurak \\
	      	HostName 156.18.40.236 \\
	  	Port 22 \\
		User hluo \\ 
		PubkeyAuthentication yes \\
	\item Usage : ssh zaurak (auto-completation) \\
		$\iff$ ssh hluo@156.18.40.236
\end{itemize}
\end{frame}

%------------------------------------------------
\begin{frame}
\frametitle{ssh acess without passwd and transfer files}
\begin{itemize}
	\item Generate and upload public key (fingerprint) of the local machine to get acess on a remote without passwd:\\
	          ssh-keygen -t rsa \\
		  ssh-copy-id zaurak
		  
	\item Copy file/dir from remote (auto-completation if public key of current machine is on the remote): \\
			scp zaurak:remote\_file local\_dir \\
			scp -r zaurak:remote\_dir local\_dir \\
			or \\
			scp local\_file zaurak:remote\_dir \\
			scp -r local\_dir zaurak:remote\_dir \\
			Remote files can be manipulated like local files by preceding the alias of remote (Ex: zaurak)
\end{itemize}
\end{frame}

%------------------------------------------------

\begin{frame}
\Huge{\centerline{The End}}
\end{frame}

%----------------------------------------------------------------------------------------

\end{document} 
